% МАКРОСЫ, СПЕЦИФИЧЕСКИЕ ДЛЯ ДАННОГО ДОКУМЕНТА   Подключается в преамбуле

\hyphenation{классицис-ти-чес-кой} % Помогаем LaTeX с переносами
\hyphenation{кирил-ли-чес-ких} % Помогаем LaTeX с переносами



\newcolumntype{L}[1]{>{\raggedright\arraybackslash}p{#1}}% Выравнивание столбца по левому краю
\newcolumntype{C}[1]{>{\centering\arraybackslash}p{#1}}% Выравнивание столбца по левому центру
\newcolumntype{R}[1]{>{\raggedleft\arraybackslash}p{#1}}% Выравнивание столбца по правому краю



\newminted[Lua]{lua}{
               fontsize=\footnotesize,
               %linenos,
               breaklines,    % Перенос кода на другую строку
               numbersep=2mm,  %Отступ от номеров строк до кода
               xleftmargin=0mm,
               frame = single, %Код заключен в прямоугольник
               baselinestretch=1.1 %Расстояние между строчками
               }

\newminted[Diff]{diff}{
               fontsize=\footnotesize,
               %linenos,
               breaklines,    % Перенос кода на другую строку
               numbersep=2mm,  %Отступ от номеров строк до кода
               xleftmargin=0mm,
               frame = single, %Код заключен в прямоугольник
               baselinestretch=1.1 %Расстояние между строчками
               }

\newminted[CcodeII]{c}{
               fontsize=\footnotesize,
               %linenos,
               breaklines,    % Перенос кода на другую строку
               numbersep=2mm,  %Отступ от номеров строк до кода
               xleftmargin=0mm,
               frame = single, %Код заключен в прямоугольник
               baselinestretch=1.1 %Расстояние между строчками
               }
               
\newminted[BASHcode]{bash}{
               fontsize=\footnotesize,
               %linenos,
               breaklines,    % Перенос кода на другую строку
               numbersep=2mm,  %Отступ от номеров строк до кода
               xleftmargin=0mm,
               frame = single, %Код заключен в прямоугольник
               framerule = 1.5pt,
               baselinestretch=1.1 %Расстояние между строчками
               }

\newminted[TEXTcode]{text}{
               fontsize=\footnotesize,
               %linenos,
               breaklines,    % Перенос кода на другую строку
               numbersep=0mm,  %Отступ от номеров строк до кода
               xleftmargin=0mm,
               frame = single, %Код заключен в прямоугольник
               baselinestretch=1.1 %Расстояние между строчками
               }

\newminted[TEXTpicture]{text}{
               fontsize=\footnotesize,
               %linenos,
               breaklines,    % Перенос кода на другую строку
               numbersep=2mm,  %Отступ от номеров строк до кода
               xleftmargin=0mm,
               frame = single, %Код заключен в прямоугольник
               baselinestretch=0.7 %Расстояние между строчками
               }

\newminted[MAKEFILEcode]{make}{
               fontsize=\footnotesize,
               %linenos,
               breaklines,    % Перенос кода на другую строку
               numbersep=2mm,  %Отступ от номеров строк до кода
               xleftmargin=0mm,
%               frame = single, %Код заключен в прямоугольник
               frame = lines,
               baselinestretch=1.1 %Расстояние между строчками 
               }

\newminted[VerilogCode]{verilog}{
               fontsize=\footnotesize,
               linenos,
               breaklines,    % Перенос кода на другую строку
               numbersep=2mm,  %Отступ от номеров строк до кода
               xleftmargin=5mm,
               frame = lines, %Линия над кодом и линия под кодом
               baselinestretch=1.1 %Расстояние между строчками
               }




\definecolor{Gray}{gray}{0.94}%
\definecolor{ColorFunction}{HTML}{DDDCF9}
\definecolor{ColorArgs}{HTML}{FBEDB4}
\definecolor{ColorRet}{HTML}{DEF8BF}






